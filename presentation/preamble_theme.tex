\documentclass[12pt,aspectratio=169]{beamer}

% usepackages
% ====================================================
\usepackage[utf8]{inputenc}
\usepackage[T1]{fontenc}
\usepackage[english]{babel}

% tables
\usepackage{xcolor}
\usepackage{tabularx}
\usepackage{colortbl}

% tikz
\usepackage{tikz}
\usepackage{pgfplots}
\usepackage{pgfplotstable}
\usepackage{tikzsymbols}

\usepgfplotslibrary{patchplots}
\usepgfplotslibrary{fillbetween}

\usetikzlibrary{shapes}
\usetikzlibrary{shapes.geometric}
\usetikzlibrary{shapes.symbols}
\usetikzlibrary{decorations.pathreplacing}
\usetikzlibrary{shadings}
\usetikzlibrary{positioning}
\usetikzlibrary{arrows}
\usetikzlibrary{trees}
\usetikzlibrary{calc}
\usetikzlibrary{arrows.meta}
\usetikzlibrary{patterns}
\usetikzlibrary{intersections}

% boxes
\usepackage[many]{tcolorbox}

% math packages and fonts
\usepackage{ccfonts}
\usepackage{eulervm}
\usepackage{amsmath}
\usepackage{amsfonts}
\usepackage{amssymb}
\usepackage{amsthm}
\usepackage{bm}
\usepackage{mathtools}
\usepackage{nicefrac}
\usepackage{slashed}
\usepackage{dsfont}
\usepackage{array}

% algorithms and listings
\usepackage[ruled,vlined,linesnumbered]{algorithm2e}
\usepackage{listings}
\usepackage{setspace}

\tcbuselibrary{listings}
\tcbuselibrary{breakable}
\tcbuselibrary{skins}

% misc
\usepackage{soul}
\usepackage{wasysym}
\usepackage{multicol}
\usepackage{hyperref}
\usepackage{animate}

% layout and theme
% ====================================================
% ====================================================

\usetheme{Copenhagen}

\definecolor{myblue1}{RGB}{35,119,189}
\definecolor{myblue2}{RGB}{95,179,238}
\definecolor{myblue3}{RGB}{129,168,207}
\definecolor{myblue4}{RGB}{26,89,142}

\setbeamercolor*{structure}{fg=myblue1,bg=blue}
\setbeamercolor*{palette primary}{use=structure,fg=white,bg=structure.fg}
\setbeamercolor*{palette secondary}{use=structure,fg=white,bg=structure.fg!75!black}
\setbeamercolor*{palette tertiary}{use=structure,fg=white,bg=structure.fg!50!black}
\setbeamercolor*{palette quaternary}{fg=black,bg=white}

\setbeamertemplate{itemize item}[circle]
\setbeamertemplate{itemize subitem}[circle]
\setbeamertemplate{itemize subsubitem}[circle]

\setbeamertemplate{enumerate item}[circle]
\setbeamertemplate{enumerate subitem}[circle]
\setbeamertemplate{enumerate subsubitem}[circle]

\setbeamercolor{itemize item}{fg=myblue1}
\setbeamercolor{itemize subitem}{fg=myblue1}
\setbeamercolor{itemize subsubitem}{fg=myblue1}

\setbeamertemplate{section in toc}[circle]
\setbeamertemplate{subsection in toc}[circle]
\setbeamerfont{subsection in toc}{size=\scriptsize}

\setbeamercolor{frametitle continuation}{fg=black}

% title graphic -- only sap logo
\titlegraphic{\includegraphics[scale=0.1]{img/tud_logo}}

% title graphic -- sap logo and dhbw logo
%\titlegraphic{\includegraphics[scale=0.1]{img/logo_sap}\hspace*{4.75cm}~%
%   	\includegraphics[scale=0.05]{img/logo_dhbw}
%}

\makeatletter
% title of a frame
\defbeamertemplate*{frametitle}{mydefault}[1][left]
{
  	\ifbeamercolorempty[bg]{frametitle}{}{\nointerlineskip}%
  	\nointerlineskip%
 	\@tempdima=\textwidth%
  	\advance\@tempdima by\beamer@leftmargin%
  	\advance\@tempdima by\beamer@rightmargin%
  	\begin{tcolorbox}[
  		enhanced,
  		outer arc=0pt,
  		arc=0pt,
  		boxrule=0pt,
  		top=0pt,
  		bottom=0pt,
  		enlarge left by=-\beamer@leftmargin,
  		enlarge right by=-\beamer@rightmargin,
  		width=\paperwidth,
  		nobeforeafter,
  		interior style={
    			left color=myblue2,
    			right color=white
    		},
  		shadow={0mm}{-0.4mm}{0mm}{black!60,opacity=0.6},    
  		shadow={0mm}{-0.8mm}{0mm}{black!40,opacity=0.4},    
  	]
    	\usebeamerfont{frametitle}%
    	\vbox{}\vskip-1ex%
    	\if@tempswa\else\csname beamer@fte#1\endcsname\fi%
    	\insertframetitle\par%
    	{%
      		\ifx\insertframesubtitle\@empty%
      		\else%
      		{\usebeamerfont{framesubtitle}\usebeamercolor[fg]{black}\insertframesubtitle\strut\par}%
      		\fi
    	}%
    	\vskip-1ex%
    	\if@tempswa\else\vskip-.3cm\fi
  	\end{tcolorbox}%
}

% footline of a frame
\defbeamertemplate*{footline}{mysplit theme}
{%
  	\leavevmode%
  	\hbox{
		\begin{beamercolorbox}[
			wd=.5\paperwidth,ht=2.5ex,dp=1.125ex,leftskip=.3cm plus1fill,rightskip=.3cm
		]{author in head/foot}%
    			\usebeamerfont{author in head/foot}\insertshortauthor\ (\insertinstitute), \insertdate
  		\end{beamercolorbox}%
  		\begin{beamercolorbox}[
			wd=.5\paperwidth,ht=2.5ex,dp=1.125ex,leftskip=.3cm,rightskip=.3cm plus1fil
		]{title in head/foot}%
    			\usebeamerfont{title in head/foot}\insertshorttitle\hfill
    			\insertprefix-\insertframenumber/\inserttotalframenumber\hspace*{0.5em}
  		\end{beamercolorbox}}%
  	\vskip0pt%
}
\makeatother

% commands and definitions
% ====================================================
% ====================================================

% page number prefix
\newcommand\insertprefix{}  % Empty by default.
\newcommand\prefix[1]{\renewcommand\insertprefix{#1}}


% listing definitions
% ====================================================
\lstdefinestyle{python}{
	language={Python},
	otherkeywords={},
	morekeywords={},
	deletekeywords={}
}

\renewcommand*\thelstnumber{\makebox[3em][r]{\ifnum\value{lstnumber}<10 0\fi\the\value{lstnumber}}}

% python linstings environment
\newtcblisting{listingPython}[1][]{
	enhanced,
	listing only,
	colback=myblue1!10!white,
	colframe=myblue1,
	overlay={
		\begin{tcbclipinterior}
			\fill[black!25] (frame.south west) rectangle ([xshift=5.1mm]frame.north west);
		\end{tcbclipinterior}
	},
	listing remove caption=false,
	left=-0.5mm,right=-1mm,
	listing options={
		style=tcblatex,
		style=python,
		keywordstyle=\bfseries\color{violet},
		commentstyle=\itshape\color{green!50!black},
		stringstyle=\color{blue},
		numbers=left,
		xleftmargin=7mm,
		tabsize=4,
	},
  	#1
}

% math definitions
% ====================================================
\DeclareMathOperator*{\argmax}{arg\,max}
\DeclareMathOperator*{\argmin}{arg\,min}
\newcommand*\diff{\mathop{}\!\mathrm{d}}

\newcommand*{\vertbar}{\rule[-1ex]{0.5pt}{2.5ex}}
\newcommand*{\horzbar}{\rule[.5ex]{2.5ex}{0.5pt}}

% algorithm definitions
% ====================================================
\newcommand\mycommfont[1]{\footnotesize\ttfamily\textcolor{myblue1}{#1}}
\SetCommentSty{mycommfont}

% allow filtering of pgfplots tables
% ====================================================
\pgfplotsset{
	discard if/.style 2 args={
        	x filter/.code={
            		\edef\tempa{\thisrow{#1}}
            		\edef\tempb{#2}
            		\ifx\tempa\tempb
                		\def\pgfmathresult{inf}
            		\fi
        	}
    	},
    	discard if not/.style 2 args={
        	x filter/.code={
            		\edef\tempa{\thisrow{#1}}
            		\edef\tempb{#2}
            		\ifx\tempa\tempb
            		\else
                		\def\pgfmathresult{inf}
            		\fi
        	}
    	},
    	colormap={customColormap}{
 		rgb255=(35,119,189)
		rgb255=(190,190,190)
        	rgb255=(255,255,255)
    	}
}

% commands
% ====================================================

% blue color box
\newtcolorbox{boxBlue}{colback=myblue1!10!white,colframe=myblue4}
\newtcolorbox{boxBlueNoFrame}{colback=myblue1!10!white,colframe=myblue1!10!white}

% highlight command
\newcommand{\highlight}[1]{\textcolor{myblue1}{\textbf{#1}}}

% divide frame into two parts
\newcommand{\divideTwo}[4]{
	\begin{minipage}{#1\textwidth}
		#2
	\end{minipage}
	\hfill
	\begin{minipage}{#3\textwidth}
		#4
	\end{minipage}
}

% divide frame into two parts (start on top)
\newcommand{\divideTwoTop}[4]{
	\begin{minipage}[t]{#1\textwidth}
		#2
	\end{minipage}
	\hfill
	\begin{minipage}[t]{#3\textwidth}
		#4
	\end{minipage}
}

% style of hyperlinks
\newcommand{\linkstyle}[1]{\underline{\texttt{#1}}}

% checkmark
\def\checkmark{\tikz\fill[scale=0.4](0,.35) -- (.25,0) -- (1,.7) -- (.25,.15) -- cycle;}

% double line circle (in tikzpicture)
\newcommand{\doublecircle}[2]{\draw[fill=white,draw=myblue1] (#1,#2) circle (2mm);\draw[fill=myblue1,draw=myblue1] (#1,#2) circle (1.5mm);}

% circled numbers
\newcommand*\circled[1]{\tikz[baseline=(char.base)]{\node[shape=circle,draw,inner sep=2pt] (char) {#1};}}

% first argument in {book, online, article}
\newcommand{\literature}[5]{
	\setbeamertemplate{bibliography item}[#1]
	\bibitem{#2}
	\highlight{#3} \\
	\textcolor{darkgray}{\textit{#4}} \\
	\textcolor{black}{#5}
}

% cite content
\newcommand{\citeAuthor}[2]{
	\vfill
	\scriptsize \textcolor{lightgray}{#1 \cite{#2}}
}

% make title page
\newcommand{\maketitlepage}{
	{
		%\usebackgroundtemplate{%
		%	\tikz[overlay,remember picture] \node[opacity=0.2, at=(current page.center)] {
  		%		\includegraphics[height=\paperheight,width=\paperwidth]{img/processor.jpg}
		%	};
		%}
		\begin{frame}[plain]
			\maketitle
		\end{frame}
	}
}

% divider page
\newcommand{\makedivider}[1]{
	{
		%\usebackgroundtemplate{%
		%	\tikz[overlay,remember picture] \node[opacity=0.2, at=(current page.center)] {
  		%		\includegraphics[height=\paperheight,width=\paperwidth]{img/processor.jpg}
		%	};
		%}
		\begin{frame}[plain]
			\vfill
			\begin{boxBlue}
				\centering
				\textbf{Section:} \\
				\large \highlight{#1}
			\end{boxBlue}
			\vfill
			\centering
			\includegraphics[scale=0.1]{img/tud_logo.png}
			\vfill
		\end{frame}
	}
}

% print overview page
\newcommand{\makeoverview}[1]{
	\begin{frame}{Lecture Overview}{}
		\begin{tabbing}
			\hspace*{3.5cm}\= \kill
			\ifnum #1=1 \highlight{\textbf{Lecture I:}} \else \textbf{Lecture I:} \fi
			\> \ifnum #1=1 \highlight{Introduction to Artificial Intelligence} \else Introduction to Artificial Intelligence \fi \\

			\ifnum #1=2 \highlight{\textbf{Lecture II:}} \else \textbf{Lecture II:} \fi
			\> \ifnum #1=2 \highlight{(Un-)Informed Search/Local Search} \else (Un-)Informed Search/Local Search \fi \\
			
			\ifnum #1=3 \highlight{\textbf{Lecture III:}} \else \textbf{Lecture III:} \fi
			\> \ifnum #1=3 \highlight{Adversarial Search} \else Adversarial Search \fi \\
			
			\ifnum #1=4 \highlight{\textbf{Lecture IV:}} \else \textbf{Lecture IV:} \fi
			\> \ifnum #1=4 \highlight{Machine Learning I -- Terminology and Optimization} \else Machine Learning I -- Terminology and Optimization \fi \\
			
			\ifnum #1=5 \highlight{\textbf{Lecture V:}} \else \textbf{Lecture V:} \fi
			\> \ifnum #1=5 \highlight{Machine Learning II -- Deep Learning (Neural Networks)} \else Machine Learning II -- Deep Learning (Neural Networks) \fi \\
			
			\ifnum #1=6 \highlight{\textbf{Lecture VI:}} \else \textbf{Lecture VI:} \fi
			\> \ifnum #1=6 \highlight{Machine Learning III -- Bayesian Learning} \else Machine Learning III -- Bayesian Learning \fi \\
			
			\ifnum #1=7 \highlight{\textbf{Lecture VII:}} \else \textbf{Lecture VII:} \fi
			\> \ifnum #1=7 \highlight{Machine Learning IV -- Reinforcement Learning} \else Machine Learning IV -- Reinforcement Learning \fi \\
			
			\ifnum #1=8 \highlight{\textbf{Lecture VIII:}} \else \textbf{Lecture VIII:} \fi
			\> \ifnum #1=8 \highlight{Machine Learning V -- Evaluation of ML Algorithms} \else Machine Learning V -- Evaluation of ML Algorithms \fi \\
			
			\ifnum #1=9 \highlight{\textbf{Lecture IX:}} \else \textbf{Lecture IX:} \fi
			\> \ifnum #1=9 \highlight{Graphical Models and Bayes Nets} \else Graphical Models and Bayes Nets \fi \\
			
			\ifnum #1=10 \highlight{\textbf{Lecture X:}} \else \textbf{Lecture X:} \fi
			\> \ifnum #1=10 \highlight{Philosophical Basics and Wrap-Up} \else Philosophical Basics and Wrap-Up \fi \\
		\end{tabbing}
	\end{frame}
}

% Print thank you page
\newcommand{\makethanks}{
	\begin{frame}[plain]
	
		\vfill
		\begin{boxBlue}
			\centering
			\Large \highlight{Thank you very much for the attention!}
		\end{boxBlue}
		
		\vfill\footnotesize
		\begin{tabbing}
			\hspace*{1.5cm}\= \kill
			\highlight{Topic:} 	\> \inserttitle \\
			\highlight{Date:} 		\> \insertdate
		\end{tabbing}
		
		\vfill
		\highlight{Contact:} \\
		\insertauthor \\
		\insertinstitute
		
		\vfill\normalsize
		\begin{center}
			\large\highlight{Do you have any questions?}
		\end{center}
		\vfill
	\end{frame}
}